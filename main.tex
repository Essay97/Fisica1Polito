\documentclass{report}

%\usepackage{fullpage}
\usepackage{amsmath}
\usepackage[makeroom]{cancel}
\usepackage{hyperref}
\usepackage{tabularx}
\usepackage[a]{esvect}
\usepackage{mathtools}
\usepackage{physics}
\usepackage{graphicx}

\hypersetup{
    colorlinks,
    citecolor=black,
    filecolor=black,
    linkcolor=black,
    urlcolor=black
}

% Definisce la funzione esponenziale
\newcommand{\me}{\mathrm{e}}

% Definisce le colonne centrate per tabularx
\newcolumntype{C}{>{\centering\arraybackslash}X}

% Definizioni dei versori (da usare in math mode)
%% Comando "astratto"
\newcommand{\versore}[1]{\hat{\mathbf{u}}_{#1}}
%% u_x
\newcommand{\ux}{\versore{x}}
%% u_y
\newcommand{\uy}{\versore{y}}
%% u_z
\newcommand{\uz}{\versore{z}}
%% u_T
\newcommand{\uT}{\versore{T}}
%% u_N
\newcommand{\uN}{\versore{N}}
%% u_r
\newcommand{\ur}{\versore{r}}
%% u_theta
\newcommand{\uth}{\versore{\theta}}

% Definisce un vettore
\newcommand{\vett}[1]{\vv{\boldsymbol{\mathbf{#1}}}}

\title{Formulario Fisica 1}
\author{Enrico Saggiorato}
\date{A.A. 2022 - 2023}

\begin{document}
\setcounter{secnumdepth}{1}
\maketitle
\tableofcontents

\part{Meccanica}

\chapter{Cinematica del punto}

\setcounter{section}{2}
\section{Velocità nel moto rettilineo}
\subsection{Velocità media}
\begin{equation*}
    v_m = \frac{\Delta x}{\Delta t} = \frac{x_2 - x_1}{t_2 - t_1}
\end{equation*}
\subsection{Velocità istantanea}
\begin{equation}
    v = \frac{dx}{dt}
\end{equation} 
\subsection{Spazio data la velocità}
Data la (1.1), si ha \(dx = v\,dt\), dunque:
\begin{align}
    \int_{x_0}^{x}dx & = \int_{t_0}^{t}v(t)\,dt \nonumber \\
    x - x_0 & = \int_{t_0}^{t}v(t)\,dt \nonumber \\
    x(t) & = x_0 + \int_{t_0}^{t}v(t)\,dt
\end{align}
\subsection{Velocità media data la velocità istantanea}
Se nella (1.2) port a sinistra \(x_0\) e divido per \(t-t_0\) entrambi i membri,
ottengo la relazione tra velocità media e velocità istantanea:
\begin{align}
    \frac{x - x_0}{t - t_0} & = \frac{\int_{t_0}^{t}v(t)\,dt}{t - t_0} \nonumber \\
    v_m & = \frac{1}{t - t_0} \int_{t_0}^{t}v(t)\,dt
\end{align}
\subsection{Moto rettilineo uniforme}
Dato che nel moto rettilineo uniforme \(v\) è costante, la (1.2) diventa:
\begin{equation}
    x(t) = x_0 + v \int_{t_0}^{t}\,dt = x_0 + v(t - t_0)
\end{equation}
Se \(t_0 = 0\)
\begin{equation*}
    x(t) = x_0 + vt
\end{equation*}

\section{Accelerazione nel moto rettilineo}
\subsection{Accelerazione istantanea}
\begin{equation}
    a = \frac{dv}{dt} = \frac{d^{2}x}{dt^2}
\end{equation}
\subsection{Velocità data l'accelerazione}
Data la (1.5) otteniamo \(dv = a\,dt\), dunque, integrando:
\begin{align}
    \int_{v_0}^{v}dv & = \int_{t_0}^{t}a(t)\,dt \nonumber \\
    v - v_0 & = \int_{t_0}^{t}a(t)\,dt \nonumber \\
    v(t) & = v_0 + \int_{t_0}^{t}a(t)\,dt
\end{align}
\subsection{Moto uniformemente accelerato: velocità}
Dato che nel moto uniformemente accelerato l'accelerazione è costante, 
la (1.6) diventa:
\begin{equation}
    v(t) = v_0 + a \int_{t_0}^{t}dt = v_0 + a(t-t_0)
\end{equation}
Se \(t_0 = 0\)
\begin{equation*}
    v(t) = v_0 + at
\end{equation*}
\subsection{Moto uniformemente accelerato: spazio}
Date la (1.2) e la (1.7), ottengo così la legge oraria del moto uniformemente accelerato:
\begin{align}
    x(t) & = x_0 + \int_{t_0}^{t}v(t)\,dt \nonumber \\
    & = x_0 + \int_{t_0}^{t}[v_0 + a(t - t_0)]\,dt \nonumber \\
    & = x_0 + v_0(t - t_0) + \frac{1}{2}a(t-t_0)^2
\end{align}
Se \(t_0 = 0\)
\begin{equation*}
    x(t) = x_0 + v_0 t + \frac{1}{2}at^2
\end{equation*}
\subsection{Velocità data la posizione}
Se è nota la dipendenza dell'accelerazione dalla posizione \(a[x(t)]\), posso
trovare la velocità \(v[x(t)]\). Dato che
\begin{align*}
    a = \frac{dv[x(t)]}{dt} = \frac{dv}{dx}\frac{dx}{dt} && \text{per la regola della catena}
\end{align*}
e che \(\frac{dx}{dt} = v\), possiamo scrivere:
\begin{equation*}
    a = v\frac{dv}{dx} \implies a\,dx = v\,dv
\end{equation*}
Integrando:
\begin{equation}
    \int_{x_0}^{x}a\,dx = \int_{v_0}^{v}v\,dv = \frac{1}{2}[v^2 - v_0^2]
    \implies \int_{x_0}^{x}a\,dx = \frac{1}{2}v^2 - \frac{1}{2}v_0^2
\end{equation}
Per il moto uniformemente accelerato:
\begin{align}
    a\int_{x_0}^{x}dx & = \frac{1}{2}v^2 - \frac{1}{2}v_0^2 \nonumber \\
    a(x - x_0) & = \frac{1}{2}v^2 - \frac{1}{2}v_0^2 \nonumber \\
    \frac{1}{2}v^2 & = \frac{1}{2}v_0^2 + a(x - x_0) \nonumber \\
    v^2 & = v_0^2 + 2a(x - x_0)
\end{align}

\section{Moto verticale di un corpo}
Le formule si deducono facilmente da quelle delle sezioni precedenti, devo ancora decidere se esplicitarne qualcuna

\section{Moto armonico semplice}
\subsection{Legge oraria}
\begin{equation}
    x(t) = A\sin(\omega t + \phi)
\end{equation}
\subsection{Periodo}
Il periodo per definzione è \(T = t' - t\) con \(x(t') = x(t)\). Le fasi nei
due istanti differiscono di \(2\pi\), dunque \(\omega t' + \phi = \omega t + \phi + 2\pi\)
Ne segue:
\begin{align}
    \omega t' - \omega t & = \bcancel{\phi} - \bcancel{\phi} + 2\pi \nonumber \\
    t' - t & = \frac{2\pi}{\omega} \nonumber \\
    T & = \frac{2\pi}{\omega} \implies \omega = \frac{2\pi}{T}
\end{align}
\subsection{Frequenza}
\begin{equation*}
    \nu = \frac{1}{T} = \frac{\omega}{2\pi}
\end{equation*}
\subsection{Velocità}
Derivando la (1.11):
\begin{equation}
    v(t) = \omega A\cos(\omega t + \phi)
\end{equation}
\subsection{Accelerazione}
Derivando la (1.13):
\begin{equation}
    a(t) = -\omega^2 A\sin(\omega t + \phi) = -\omega^2 x
\end{equation}
Questa formula definisce anche la relazione tra accelerazione e spazio.
\subsection{Velocità data la posizione}
Data la (1.9), posso sostituire la (1.14) per trovare:
\begin{align*}
    \int_{x_0}^{x} -\omega^2 x\,dx & = \frac{1}{2}v^2 - \frac{1}{2}v_0^2 \\
    -\omega^2\cdot\frac{1}{2}\left(x^2 - x_0^2\right) & = \frac{1}{2}v^2 - \frac{1}{2}v_0^2 \\
    \cancel{\frac{1}{2}}\omega^2\left(x_0^2 - x^2\right) & = \cancel{\frac{1}{2}}v^2 - \cancel{\frac{1}{2}}v_0^2 \\
    v^2 & = \omega^2\left(x_0^2 - x^2\right) + v_0^2 
\end{align*}
Al centro del periodo, dove la velocità è massima e lo spazio è minimo, (\(x_0 = 0\) e \(v_0 = \omega A \)):
\begin{align}
    v^{2}(x) & = \omega^2\left(0 - x^2\right) + (\omega A)^2 \nonumber \\
    & = \omega^2\left(A^2 - x^2\right)
\end{align}
\subsection{Equazione differenziale del moto armonico}
\begin{equation}
    \frac{d^{2}x}{dt^2} + \omega^2 x = 0
\end{equation}

\section{Moto rettilineo smorzato esponenzialmente}
Un moto vario è smorzato esponenzialmente se per esso vale l'equazione differenziale
\begin{equation*}
    \frac{dv}{dt} = -kv
\end{equation*}
\subsection{Velocità}
Posso separare le variabili dell'equazione differenziale così \(\frac{dv}{v} = -k\,dt\) e integrare:
\begin{align*}
    \int_{v_0}^{v}\frac{1}{v}\,dv & = -k \int_{0}^{t}dt \\
    \log v - \log v_0 & = -kt \\
    \log\frac{v}{v_0} & = -kt
\end{align*}
Passando all'esponenziale:
\begin{equation*}
    \frac{v}{v_0} = \me^{-kt} \implies v = v_0 \me^{-kt}
\end{equation*}
\subsection{Velocità data la posizione}
Per avere la relazione tra velocità e posizione devo prima trovare la 
relazione tra accelerazione e posizione:
\begin{equation*}
    a = \frac{d}{dt}\biggl(v[x(t)]\biggr) = \frac{dv}{dx}\frac{dx}{dt} = v\frac{dv}{dx} = -kv \implies \frac{dv}{dx} = -k
\end{equation*}
Riscrivendo come \(dv = -k\,dx\) e integrando:
\begin{align*}
    \int_{v}^{v_0}dv & = -k\int_{0}^{x}dx \\
    v - v_0 & = -kx \\
    v(x) & = v_0 - kx
\end{align*}
\subsection{Legge oraria}
Integrando la velocità:
\begin{align*}
    x(t) & = x_0 + \int_{0}^{t}v(t)\,dt = \int_{0}^{t}v_0 \me^{-kt} 
    = -\frac{v_0}{k}\left(\me^{-kt} - \cancelto{1}{\me^{-k\cdot 0}}\right) \\
    & = \frac{v_0}{k}\left(1-\me^{-kt}\right)
\end{align*}

% Saltare 1.8 Paradosso di Zenone
\stepcounter{section}

\section{Moto nel piano, posizione e velocità}
\subsection{Relazioni tra coordinate cartesiane e polari}
\setlength{\extrarowheight}{20pt}
\begin{tabularx}{\textwidth}{CC}  
    \(x = r\cos\theta\) & \(y = r\sin\theta\) \\
    \(r = \sqrt{x^2 + y^2}\) & \(\tan\theta = \frac{y}{x}\)
\end{tabularx}

\subsection{Raggio vettore}
Se \(O\) è l'origine degli assi cartesiani:
\begin{equation*}
    \vett{r}(t) = \vett{OP} = x(t)\ux + y(t)\uy
\end{equation*}

\subsection{Velocità vettoriale}
\begin{equation}
    \vett{v} = \frac{d\vett{r}}{dt}
\end{equation}
Esplicitando il versore:
\begin{equation}
    \vett{v} = \frac{ds}{dt}\uT
\end{equation}

\subsection{Componenti cartesiane della velocità}
Dato che il raggio vettore in coordinate cartesiane è \(\vett{r} = x\ux + y\uy\):
\begin{align*}
    \vett{v} & = \frac{d\vett{r}}{dt} 
    = \frac{dx}{dt}\ux + x\frac{d\ux}{dt} + \frac{dy}{dt}\uy + y\frac{d\uy}{dt}
    \intertext{I versori non variano nel tempo e le loro derivate valgono 0}
    & = \frac{dx}{dt}\ux + \frac{dy}{dt}\uy \\
    & = v_x\ux + v_y\uy
\end{align*}

\subsection{Componenti polari della velocità}
Dato che il raggio vettore in coordinate polari è \(\vett{r} = r\ur\):
\begin{align}
    \vett{v} & = \frac{d\vett{r}}{dt} = \frac{dr}{dt}\ur + r\frac{d\ur}{dt} \nonumber \\
    & = \frac{dr}{dt}\ur + r\frac{d\theta}{dt}\uth 
\end{align}
Le due componenti si chiamano rispettivamente \emph{velocità radiale} e \emph{velocità trasversa}.
E il modulo vale
\begin{equation*}
    \norm{\vett{v}} = \sqrt{\left(\frac{dr}{dt}\right)^2 + r^2\left(\frac{d\theta}{dt}\right)^2}
\end{equation*}

\subsection{Posizione data la velocità}
\begin{equation}
    \vett{r}(t) = \vett{r}(t_0) + \int_{t_0}^{t} \vett{v}(t)\,dt 
\end{equation}

\section{Accelerazione nel moto piano}
\subsection{Accelerazione nel piano}
\begin{equation}
    \vett{a} = \frac{d\vett{v}}{dt} = \frac{d^2\vett{r}}{dt^2}
\end{equation}
Sostituendo la (1.18):
\begin{equation*}
    \vett{a} = \frac{d}{dt}(v\uT) = \frac{dv}{dt}\uT + v\frac{d\uT}{dt}
    = \frac{dv}{dt}\uT + v\frac{d\phi}{dt}\uN
\end{equation*}
Stabilita la circonferenza osculatrice con centro in \(C\) e raggio \(R\), 
l'arco di circonferenza sotteso da \(d\phi\) è \(ds = R\,d\phi\). Riscrivo 
dunque \(\frac{d\phi}{dt}\) nel seguente modo:
\begin{align*}
    & ds = R\,d\phi \implies \frac{d\phi}{ds} = \frac{1}{R} \\
    & \implies \frac{d\phi}{dt} = \frac{d\phi}{ds}\frac{ds}{dt} = \frac{1}{R}v
\end{align*}
Sostituendo nella relazione trovata prima:
\begin{equation}
    \vett{a} = \frac{dv}{dt}\uT + \frac{v^2}{R}\uN
\end{equation}
Le due componenti si chiamano rispettivamente \emph{accelerazione tangenziale} e \emph{accelerazione centripeta}.

\subsection{Componenti cartesiane dell'accelerazione}
\begin{align*}
    \vett{a} & = \frac{d\vett{v}}{dt} 
    = \frac{d}{dt}\left(\frac{dx}{dt}\ux + \frac{dy}{dt}\uy\right) \\
    & = \frac{d^2 x}{dt^2}\ux + \cancel{\frac{dx}{dt}\frac{d\ux}{dt}}
    + \frac{d^2 y}{dt^2}\uy + \cancel{\frac{dy}{dt}\frac{d\uy}{dt}} \\
    & = a_x\ux + a_y\uy
\end{align*}

\subsection{Componenti polari dell'accelerazione}
\begin{align}
    \vett{a} & = \frac{d\vett{v}}{dt} 
    = \frac{d}{dt}\left(\frac{dr}{dt}\ur + r\frac{d\theta}{dt}\uth\right) \nonumber \\
    & = \frac{d^2 r}{dt^2}\ur + \frac{dr}{dt}\frac{d\ur}{dt} 
    + \frac{dr}{dt}\frac{d\theta}{dt}\uth + r\frac{d^2 \theta}{dt^2}\uth
    + r\frac{d\theta}{dt}\frac{d\uth}{dt} \nonumber \\
    & = \frac{d^2 r}{dt^2}\ur + \frac{dr}{dt}\frac{d\theta}{dt}\uth 
    + \frac{dr}{dt}\frac{d\theta}{dt}\uth + r\frac{d^2 \theta}{dt^2}\uth
    - r\left(\frac{d\theta}{dt}\right)^2\ur \nonumber \\
    & = \left[\frac{d^2 r}{dt^2} - r\left(\frac{d\theta}{dt}\right)^2\right]\ur
    + \left[2\frac{dr}{dt}\frac{d\theta}{dt} + r\frac{d^2 \theta}{dt^2}\right]\uth
\end{align}

\subsection{Velocità data l'accelerazione}
\begin{equation}
    \vett{v}(t) = \vett{v}(t_0) + \int_{t_0}^{t}\vett{a}(t)\,dt
\end{equation}

\section{Moto circolare}
\subsection{Velocità angolare}
La funzione dell'angolo nel tempo è definita da \(\theta(t) = \frac{s(t)}{R}\), dunque la velocità angolare è:
\begin{equation}
    \omega = \frac{d\theta}{dt} = \frac{1}{R}\frac{ds}{dt} = \frac{v}{R}
\end{equation}
Da cui anche
\begin{equation*}
    v = R\omega
\end{equation*}

\subsection{Moto circolare uniforme: leggi orarie}
\begin{align*}
    s(t) = s_0 + vt && s = s_0 \quad\text{per}\quad t = 0 
\end{align*}
\begin{align*}
    \theta(t) = \theta_0 + \omega t && \theta = \theta_0 \quad\text{per}\quad t = 0
\end{align*}

\subsection{Moto circolare uniforme: accelerazione}
% Riga spezzata per evitare overfull hbox
Nel moto circolare uniforme è presente la componente tangenziale \\dell'accelerazione,
per cui, dalla (1.22):
\begin{equation}
    a = a_N = \frac{v^2}{R}
\end{equation}

\subsection{Moto circolare uniforme: scomposizione lungo gli assi}
Il moto circolare uniforme si scompone in due moti armonici:
\begin{equation*}
    x(t) = R\cos(\omega t + \theta_0)
\end{equation*}
\begin{equation*}
    y(t) = R\sin(\omega t + \theta_0)
\end{equation*}

\subsection{Accelerazione angolare}
Se il moto non è uniforme l'accelerazione ha anche una componente tangenziale 
\(a_T = \frac{dv}{dt}\). Anche \(\omega\) varia:
\begin{equation}
    \alpha = \frac{d\omega}{dt} = \frac{d^2\theta}{dt^2} 
    = \frac{d^2}{dt^2}\left(\frac{s(t)}{R}\right)
    = \frac{1}{R}\frac{dv}{dt} = \frac{a_T}{R}
\end{equation}

\subsection{Velocità angolare data l'accelerazione angolare}
\begin{equation}
    \omega = \omega_0 + \int_{t_0}^{t}\alpha(t)\,dt
\end{equation}

\subsection{Angolo data la velocità angolare}
\begin{equation}
    \theta = \theta_0 + \int_{t_0}^{t}\omega(t)\,dt
\end{equation}

\subsection{Moto circolare uniformemente accelerato}
In questo moto l'accelerazione è costante. quindi sono costanti \(\alpha\)
e \(a_T\). Dalle (1.28) e (1.29) con \(t_0 = 0\):
\begin{equation*}
    \omega = \omega_0 + \alpha t
\end{equation*}
\begin{equation*}
    \theta = \theta_0 + \int_{0}^{t}(\omega_0 + \alpha t)\,dt
    = \theta_0 + \omega_0 t + \frac{1}{2}\alpha t^2
\end{equation*}

% Salta la relazione tra velocità angolare e accelerazione dato l'angolo, si ottiene nell solito modo 
% con la regola della catena
\stepcounter{equation}

\subsection{Notazione vettoriale: velocità}
\begin{equation}
    \vett{v} = \vett{\omega} \times \vett{r}
\end{equation}
Con \(\norm{\vett{\omega}} = \frac{d\theta}{dt}\)

\subsection{Notazione vettoriale: accelerazione}
Derivando \(\vett{\omega}\) si ottiene \(\vett{\alpha}\), un vettore parallelo 
a \(\vett{\omega}\) che ha modulo \(\norm{\vett{\alpha}}  = \frac{d\omega}{dt}\).
Tramite queste si esprime l'accelerazione nel moto circolare:
\begin{equation}
    \vett{a} = \frac{d\vett{v}}{dt} 
    = \frac{d}{dt}\left(\vett{\omega} \times \vett{r}\right)
    = \frac{d\vett{\omega}}{dt} \times \vett{r} + \vett{\omega} \times \frac{d\vett{r}}{dt}
    = \vett{\alpha} \times \vett{r} + \vett{\omega} \times \vett{v}
\end{equation} 
Con \(\vett{\alpha} \times \vett{r}\) \emph{accelerazione tangenziale} \(\vett{a}_T\)
e \(\vett{\omega} \times \vett{v}\) \emph{accelerazione centripeta}.

% Salta il moto di precessione
\stepcounter{equation}

\section{Moto parabolico dei corpi}
\subsection{Condizioni iniziali}
L'analisi di seguito presuppone la presenza della costante accelerazione di gravità
\(\vett{a} = \vett{g} = -g\uy\) e, al tempo \(t = 0\):
\begin{align*}
    \vett{r} & = 0 \\
    \vett{v} & = \vett{v}_0
\end{align*}
Chiameremo \(\alpha\) l'angolo tra la velocità iniziale \(\vett{v}_0\)
e l'asse delle ascisse.

\subsection{Moto parabolico: velocità}
Data la (1.24), si può scrivere:
\begin{equation*}
    \vett{v}(t) = \vett{v}(0) + \int_{0}^{t}\vett{a}(t)\,dt  
    = \vett{v}_0 -gt\uy
\end{equation*}
Scomponendo \(\vett{v}_0\) nelle sue componenti cartesiane:
\begin{equation*}
    \vett{v}(t) = v_0\cos\alpha\ux + v_0\sin\alpha\uy - gt\uy 
    = v_0\cos\alpha\ux + (v_0\sin\alpha - gt)\uy
\end{equation*}
Tale scomposizione individua la velocità dei moti proiettati sui due assi, 
dunque le leggi orarie sono:

\setlength{\extrarowheight}{20pt}
\begin{tabularx}{\textwidth}{CC} 
    \(x = v_0\cos\alpha t\) & \(y = v_0\sin\alpha t - \dfrac{1}{2}gt^2\)
\end{tabularx}

\subsection{Moto parabolico: traiettoria}
Ricavando dalla legge oraria della proiezione del moto sull'asse \(x\)
il tempo\\ \(t = \dfrac{x}{v_0\cos\alpha}\), sostituisco nella legge oraria 
di \(y\) e trovo:
\begin{equation*}
    y(x) = x\tan\alpha - \frac{g}{2v_0^2\cos^2\alpha}x^2
\end{equation*}

\subsection{Moto parabolico: gittata}
Trovo la gittata dalla traiettoria imponendo \(y(x) = 0\) e esplicitando \(x\):
\begin{equation*}
    x_G = \frac{2v_0^2\cos\alpha\sin\alpha}{g} = 2x_M
\end{equation*}
dove \(x_M\) è la coordinata del punto centrale della gittata e quindi, 
per simmetria della parabola, è il punto di massima altezza. 

\subsection{Moto parabolico: altezza massima}
L'altezza massima dunque si trova tramite la traiettoria calcolando \(y(x_M)\):
\begin{equation*}
    y(x_M) = y_M = \frac{v_0^2\sin^2\alpha}{2g}
\end{equation*}

\section{Moto nello spazio. Composizione di moti}
\subsection{Posizione in coordinate cartesiane}
\begin{equation*}
    \vett{r}(t) = x(t)\ux + y(t)\uy + z(t)\uz
\end{equation*}

\subsection{Velocità in coordinate cartesiane}
\begin{equation*}
    \vett{v}(t) = \frac{d\vett{r}}{dt} 
    = \frac{dx}{dt}\ux + \frac{dy}{dt}\uy + \frac{dz}{dt}\uz
    = v_x\ux + v_y\uy + v_z\uz
\end{equation*}

\subsection{Accelerazione in coordinate cartesiane}
\begin{equation*}
    \vett{a} = \frac{d\vett{v}}{dt} 
    = \frac{d^2x}{dt}\ux + \frac{d^2y}{dt}\uy + \frac{d^2z}{dt}\uz
    = a_x\ux + a_y\uy + a_z\uz
\end{equation*}

% Da rivedere la composizione di moti

% Salta il riepilogo
\stepcounter{section}



\part{Termodinamica}

\end{document}